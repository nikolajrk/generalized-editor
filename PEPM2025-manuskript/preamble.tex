\usepackage[utf8]{inputenc}
%\usepackage{lmodern}
%\usepackage{uppunctlm}
\usepackage{amsmath}
\usepackage{array, caption, booktabs}
\usepackage{stmaryrd}
\usepackage{graphicx} % Required for inserting images
\usepackage{semantic}
\usepackage{mathtools}
\usepackage{float}
\usepackage{xspace}
%\usepackage{fancyvrb}
\usepackage{listings}
\usepackage{thmtools}
\usepackage[capitalise,noabbrev]{cleveref}
\usepackage{subcaption}

%\usepackage[
%backend=biber,
%style=numeric,
%sorting=ynt,
%maxcitenames=1,
%maxbibnames=99
%]{biblatex}
%\addbibresource{bibliography.bib}

\newcolumntype{L}{>{$}l<{$}} % left-oriented math column type
\newcolumntype{R}{>{$}r<{$}} % right-oriented math column type
\newcolumntype{C}{>{$}c<{$}} % center-oriented math column type

\newcommand{\runa}[1]{\ensuremath{{\normalsize{\textsc{(#1)}}}\;\;}}

\newcommand{\lr}[1]{\langle #1 \rangle} % config macro: <#1>
\newcommand{\sub}[1]{\/\textsubscript{#1}} % subscript macro
\newcommand{\ltrans}[1]{\stackrel{#1}{\Rightarrow}} % labelled transition pil
\newcommand{\ltransn}[1]{\stackrel{#1}{\nRightarrow}} % labelled transition pil
\newcommand{\infeshort}[1]{\inference[\runa{#1}]} %formatting of \inference, when in a table side by side
\newcommand{\infelong}[1]{\inference[\runa{#1}]} %formatting of \inference, when on a single line
\newcommand{\define}{\stackrel{\text{def}}{=}} %def =
\newcommand{\encode}[1]{\llbracket #1 \rrbracket} %encoding brackets
\newcommand{\match}[1]{\{\mid #1 \mid\}} %encoding brackets
\newcommand{\child}{\textsf{child }} % 'child' stands out
\newcommand{\term}{$\lambda$-term\xspace} % lambda term
\newcommand{\OP}{\mathcal{O}} %family of operators
\newcommand{\OPc}{\hat{\mathcal{O}}} %cursorless family of operators
\newcommand{\OPcc}{\dot{\mathcal{O}}} %cursorless family of operators
\newcommand{\SORT}{\mathcal{S}} %set of sorts
\newcommand{\SORTc}{\hat{\mathcal{S}}} %set of cursorless sorts
\newcommand{\SORTcc}{\dot{\mathcal{S}}} %set of cursorless sorts
\newcommand{\VAR}{\mathcal{X}} %family of variables
\newcommand{\VARc}{\hat{\mathcal{X}}} %family of variables
\newcommand{\VARcc}{\dot{\mathcal{X}}} %family of variables
\newcommand{\AST}{\mathcal{A}[\VAR]} %family of asts
\newcommand{\ABT}[1]{\mathcal{B}[\VAR #1]} %family of abts
\newcommand{\ABTc}[1]{\hat{\mathcal{B}}[\VARc #1]} %family of cursorless abts
\newcommand{\ABTcc}[1]{\dot{\mathcal{B}}[\VARcc #1]} %family of cursorless abts
\newcommand{\alphac}{\rightarrow_\alpha} %alpha-conversion
\newcommand{\betar}{\rightarrow_\beta} %beta-reduction
\newcommand{\alphae}{\equiv_\alpha} %alpha-equivalence
\newcommand{\betae}{\equiv_\beta} %beta-equivalence
\newcommand{\stlc}{\ensuremath{\lambda_{->}}\xspace} %simply typed lambda calculus
\newcommand{\hole}[1]{\llparenthesis \ \rrparenthesis_#1} %hole, subscripted with a sort
\newcommand{\ctxhole}{[ \, \cdot \, ]} %context hole
\newcommand{\fix}{\text{fix}\xspace} %correct fix

\renewcommand{\vec}[1]{\overrightarrow{#1}}

\newcommand{\federeboks}[3]{
    \begin{figure}[H]
    \vspace{-2.9mm}
    \centering
    %\fbox
    #3
    %\vspace{-2mm}
    \end{figure}
}

\newcommand{\skat}[1]{\ensuremath{\textbf{#1}}}
\newcommand{\Edt}{\skat{Edt}}
\newcommand{\Var}{\skat{Var}}
\newcommand{\Types}{\skat{Types}}

\newcommand{\pra}{\hookrightarrow}

\newcommand{\bs}{$\textbackslash$}

\lstdefinelanguage{elm}
{
    morekeywords={
            alias,
            as,
            case,
            else,
            exposing,
            if,
            import,
            in,
            let,
            module,
            of,
            port,
            then,
            type,
            where
        },
    sensitive=true, % Keywords are case-sensitive
    morecomment=[s]{\{-}{-\}}, % s is for start and end delimiter
    morecomment=[l]{--},
    morestring=[b]" % Defines that strings are enclosed in double quotes
}


\lstdefinestyle{inline}{
    columns=fullflexible,
    showspaces=false,
    showtabs=false,
    breaklines=true,
    showstringspaces=false,
    breakatwhitespace=true,
    escapeinside={(*@}{@*)},
    % commentstyle=\color{greencomments},
    % keywordstyle=\color{bluekeywords},
    % stringstyle=\color{redstrings},
    % numberstyle=\color{graynumbers},
    % numberstyle=\tiny\color{codegray},
    % backgroundcolor=\color{myinlinecolorback},
    basicstyle=\ttfamily\footnotesize,
    frame=single,
    frameround=tttt,
    tabsize=2,
    captionpos=b,
    numbers=none,
    % xleftmargin=2em,
}

\lstdefinestyle{examplestyle}{
    % backgroundcolor=\color{myexamplecolorback},
    % commentstyle=\color{codegreen},
    % keywordstyle=\color{magenta},
    % numberstyle=\tiny\color{codegray},
    % stringstyle=\color{codepurple},
    basicstyle=\ttfamily\footnotesize,
    breakatwhitespace=false,
    breaklines=true,
    captionpos=b,
    tabsize=2,
    frame=single,
    numbers=none,
    % xleftmargin=2em,
    frameround=tttt,
    % framexleftmargin=1.5em
}

\lstset{style=inline}

%%% Local Variables:
%%% mode: latex
%%% TeX-master: "Article"
%%% End:
